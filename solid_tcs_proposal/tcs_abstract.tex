We propose to measure exclusive $e^+e^-$ production with the SoLID detector
using an 11 GeV polarized beam and a $LH_2$ target to study the reaction
$\gamma p \to \gamma^* p^\prime \to e^+ e^- p^\prime$, known as Timelike
Compton Scattering (TCS), which is the timelike equivalent of (spacelike)
DVCS. Both the differential cross section and moments of the weighted cross
section will be measured as a function of the four-momentum transfer $-t$,
the outgoing photon virtuality $Q^{\prime 2}$ (up to 9 GeV$^2$), and the
skewness $\eta$. The latter reflects the difference between the initial
and final momentum fraction carried by the struck quark, and corresponds
to $\xi$ in DVCS.
TCS is sensitive to the real part of the Compton form factors and
provides access to GPD components describing the distribution of
matter in the nucleon (form factors of energy-momentum tensor,
$D$-term), complementing the information obtained from DVCS.

The high luminosity of SoLID will make it possible to perform a mapping
of the $Q^{\prime 2}$- and $\eta$-dependence, which is essential for
understanding factorization, higher-twist effects, and NLO corrections.
This proposed experiment is complementary to the approved CLAS12 experiment
E12-12-001~\cite{E12-12-001}, which will measure the $t$-dependence in
wider bins of $Q^{\prime 2}$ and $\eta$, but will not collect sufficient
statistics for the proposed $Q^{\prime 2}$ and $\eta$ studies.
The CLAS12 and SoLID experiments are further mutually supportive in
that performing this new kind of measurement using two detector setups,
each with a different acceptance, will provide an essential cross check.
It could also result in reduced overall systematic uncertainties on,
for instance, the real part of the Compton form factor $\mathcal{H}$,
to which TCS provides a straightforward access.
The proposed experiment can run in parallel with experiment E12-12-006
\cite{E12-12-006}, which has been approved for 60 days. The projections
are thus shown for an effective 50 days of running. However, since TCS
experiments at 12 GeV are always statistics limited, we will also
consider asking for additional beam time.
