This proposed experiment will require a longitudinally polarized ($>80$\%) 11
GeV beam and an unpolarized proton target in the SoLID detector.
For photoproduction using quasi-real photons, the detector will need a recoil
baryon detection and identification capability. This could be most easily
accomplished by extending the high-resolution time-of-flight system to cover
the same angular range as the calorimeter.
Furthermore, specifically for $e^+e^-$ photoproduction, the trigger has to be
set up for a coincidence of two leptons, with the additional condition that
at least one of them produces a signal in the Cherenkov. Events where both
leptons hit the outer part of the calorimeter where there is no Cherenkov
coverage would not be useful for offline analysis due to the low pion pair
suppression factor. If it would desirable to reduce the trigger rate further,
a condition of a third particle could be added, even though this would make
it more difficult to evaluate the trigger efficiency. Alternatively, a more
open trigger satisfying the conditions above would also be a good solution.
With the provisions above, the beam-time can, in part or in full, be shared
with already approved experiments, such as E12-12-006~\cite{E12-12-006}.

