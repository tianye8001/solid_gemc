Understanding the structure and interactions of hadrons on the basis of
Quantum Chromodynamics (QCD) is one of the main objectives of nuclear physics.
The combination of fundamental properties of QCD as a quantum field theory,
such as relativity and causality, with factorization theorems allows us to
systematically explore the partonic structure of hadrons through various
processes using different probes. In this context, the correspondence between
spacelike and timelike processes plays a unique role. 

Let us consider the Drell-Yan process, $h{\bar h} \to \gamma^{\ast}X
\to l{\bar l}X$, where $\gamma^{\ast}$ has a timelike virtuality ($Q^2 > 0$),
$h$ (${\bar h}$) denotes a baryon (antibaryon), and $l$ (${\bar l}$) a lepton
(antilepton). This reaction provides important information on (anti)quark
distributions in the hadrons $h$ and ${\bar h}$.
The same distributions are, however, also probed in inclusive deep inelastic
scattering (DIS), $l h \to l^{\prime} X$, mediated by an exchange of a
spacelike virtual photon ($Q^2 < 0$), $\gamma^{\ast}h \to X$.
A comparison of the Drell-Yan and DIS results thus convincingly
demonstrated the universality of parton distribution functions (PDFs).
In this proposal we focus on the correspondence between timelike and spacelike
deeply virtual Compton scattering (DVCS), where the former is also known as
timelike Compton scattering (TCS), and the universality of generalized parton
distributions (GPDs), measured in hard exclusive processes.

In the last 15 years, hard exclusive processes have emerged as a class of
reactions providing novel information on the quark and gluon distributions
in hadrons. This information is more complete than what can be obtained
from inclusive and elastic scattering alone; for reviews, see
Refs.~\cite{Goeke:2001tz,Diehl:2003ny,Belitsky:2005qn}.
QCD factorization theorems \cite{Collins:1996fb,Collins:1998be} make it
possible to express amplitudes of hard exclusive processes in terms of GPDs,
which are expected to provide a universal (process-independent) description
of the nucleon, and have a known QCD ($Q^2$) evolution. GPDs are hybrid
distributions that combine aspects of the usual collinear PDFs, elastic
form factors, and distribution amplitudes.
As such, GPDs simultaneously encode information on parton
distributions and correlations in both momentum (in the longitudinal
direction) and coordinate (in the transverse direction) spaces.
Another interesting aspect of GPDs is their connection to the form factors
of the energy-momentum tensor, which, among other things, establishes the
decomposition of the proton spin in terms of the quark and gluon
contributions to the total orbital momentum~\cite{Ji:1996ek}.

The best studied hard exclusive process is DVCS,
$\gamma^{\ast}p \to \gamma p$, where the initial-state virtual photon is
spacelike ($Q^2 < 0$), and the final-state photon is real. From a theoretical
point of view, it is the simplest and cleanest way to access GPDs.
The leading-twist formalism is well established for DVCS at the leading and
next-to-leading orders in the strong coupling constant, and power-suppressed
corrections have been analyzed and estimated. On the experimental side, early
data have demonstrated the feasibility of DVCS measurements, established the
reaction mechanism based on the leading-twist approach (the handbag
mechanism), and provided first glimpses of the Compton form factors (CFFs)
and the related GPDs. The goal of determining the valence quark GPDs in the
nucleon through measurements of DVCS and other hard exclusive processes is
now a cornerstone of the 12 GeV program at Jefferson Lab.

Recently, a promising opportunity has emerged for extending our understanding
of GPDs by studying the timelike equivalent of traditional, spacelike DVCS.
The process, $\gamma p \to \gamma^{\ast} p$, is known as timelike Compton
scattering (TCS). Here, the timelike final-state photon immediately decays
into a lepton pair, the invariant mass of which is a measure of the photon
virtuality ($Q^{\prime 2} > 0$), and provides the hard scale for the reaction.
The leading-twist formalism for TCS~\cite{Berger:2001xd} (the factorization
theorem, the handbag reaction mechanism, etc) is as well established as that
for DVCS. However, as also shown in Ref.~\cite{Berger:2001xd}, the
phenomenology of TCS is quite different from DVCS. With an unpolarized photon
beam, TCS offers straightforward access to the real part of the CFFs through
the interference between the Compton and Bethe-Heitler (BH) amplitudes, which
can be extracted in a model-independent way from the azimuthal angular
distribution of the lepton pair into which the timelike photon decays.
Circular photon polarization also gives access to the imaginary part of CFFs.
In summary, the main motivation to study TCS includes:
\begin{itemize}
\item
A measurement of TCS will make it possible to test the universality of GPDs
implied by factorization through the timelike-spacelike correspondence with
DVCS.
\item
The straightforward access in TCS to, in particular, the real part of the
CFFs impacts models and parametrizations of GPDs in a broad range of
kinematics (light-cone fractions $\tau$ and $\eta$, which are the equivalent
of $x$ and $\xi$ in DVCS).
\item
The differential cross section (for TCS and its interference with BH) can
provide important input for global fits of CFFs
\cite{Guidal:2013rya,Guidal:2008ie}.
\end{itemize}  

However, a solid interpretation of the results from the TCS program, will also
require understanding of higher-twist and NLO corrections (in $\alpha_s$).
A general framework for the calculation of kinematic higher-twist corrections,
proportional to $|t|/Q^{\prime 2}$ and $M^2/Q^{\prime 2}$, has recently been
developed  for hard exclusive reactions~\cite{Braun:2011zr}.
The formalism was applied to DVCS~\cite{Braun:2012hq} and it was found the
higher twist corrections are important for $Q^{2}=1-10$ GeV$^2$,
\textit{i.e.}, for the kinematics largely overlapping with that of Jefferson
Lab at 6 and 12 GeV. Thus, if one wants to study factorization and effects 
related to higher-twist contributions, it is crucial to be able to map out
the full range in $Q^{\prime 2}$ with sufficient statistics.

Recent calculations~\cite{Moutarde:2013qs} suggest that the NLO corrections
may be sizeable, and larger for TCS than DVCS. They are, however, expected to
be small at large values of the skewness $\eta$, corresponding to a large
difference between the initial and final momentum fraction carried by the
struck quark. They then increase rapidly as $\eta$ goes from 0.4 to 0.1, which
is at the lower limit of the reach of 12 GeV kinematics. The expression for
$\eta$ is
\begin{equation}
\eta = - \frac{(q-q^\prime) \cdot (q+q^\prime)}{(p+p^\prime) \cdot (q+q^\prime)} = \frac{Q^{\prime 2}}{2(s - M^2) - Q^{\prime 2} + t} = \frac{Q^{\prime 2}}{4ME_\gamma - Q^{\prime 2} + t},
\label{eq:etatauQ2}
\end{equation}
where $q$, $q^\prime$, $p$, and $p^\prime$ are defined in Eq.~\ref{eq:TCS},
$M$ is the proton mass, and $E_\gamma$ is the incident photon energy.
As in DVCS, we have that $\eta \approx \tau / (2 - \tau)$, where $\tau$ is
the TCS equivalent of Bjorken $x$.
Since large values of $\eta$ correspond to large $Q^{\prime 2}$, doing the
mapping in $\eta$ requires sufficient statistics at high values of
$Q^{\prime 2}$, where the cross section is small.
Thus, in 12 GeV kinematics the region of high $Q^{\prime 2}$, where
both higher-twist and NLO corrections are expected to be small, provides a
natural reference point.
On the other hand, the NLO corrections are almost entirely due to gluons. If
they turn out to be significant at lower values of $\eta$, and since they are
predicted to be larger in TCS than DVCS, TCS could become a very interesting
new tool for studying gluons at 12 GeV.

The primary goal of this proposed experiment for SoLID is to make a precision
study of the $Q^{\prime 2}$- and $\eta$-dependence of the differential cross
section and moments of the weighted cross section for the full range of
$Q^{\prime 2}$, for which the high luminosity of SoLID is essential. This
proposal is thus complementary to the approved CLAS12 experiment E12-12-001
\cite{E12-12-001}, which will focus on studying the $t$-dependence in larger
bins of $Q^{\prime 2}$ and $\eta$.
The two detectors also offer complementary capabilities. In particular, the
SoLID detector, being based on a solenoidal magnet, has a more uniform
acceptance in the azimuthal angle $\varphi$ than CLAS12, but has a gap in
the $\vartheta$-coverage between the inner (forward) and outer detectors.
Performing this new kind of measurement using two setups, each with a
different acceptance, will not only provide an essential cross check, but
could result in reduced overall systematic uncertainties on, for instance,
the real part of the Compton form factor $\mathcal{H}$.

The feasibility of the experimental techniques involved in the measurement,
including the use of quasi-real photons (with $Q^2 < 0.1$ GeV$^2$) tagged by
detecting the complete final state except for the beam electron, have been
demonstrated in the analysis of CLAS 6 GeV data, which include pilot studies
of TCS. In terms of experimental requirements, photoproduction measurements in
SoLID will require time-of-flight detectors covering both the inner (forward)
as well as the outer calorimeter. The trigger for the reaction will have to
include at least two leptons and could require an additional track using the
time-of-flight rather than Cherenkov detector.
We thus propose to measure exclusive $e^+e^-$ production using the SoLID
detector and an 11 GeV linearly polarized electron beam and a $LH_2$ target
to study TCS over a wide range of $Q^{\prime 2}$, $\eta$, and $t$. Both the
differential cross section and the cosine and sine moments of the weighted
cross section will be measured.
