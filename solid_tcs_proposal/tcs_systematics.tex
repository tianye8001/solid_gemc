The two main sources of systematic uncertainty for the proposed measurement
are acceptance corrections and lepton identification. The former can be
expected to comparable to estimates for cross section measurements with
CLAS12, \textit{i.e.}, to be at least of the order of 5\%. As discussed in
Sec.~\ref{sec:acc}, the acceptance studies will be performed through GEMC
simulations -- the standard GEANT4 package for SoLID.

As described in Sec.~\ref{sec:tcs_selection}, lepton identification will be
performed using the Cherenkov counters (CC) and the electromagnetic
calorimeters (EC). All events used for the analysis of this proposed
experiment will have both leptons detected in one of the ECs, and at least
within the (angular and momentum) acceptance of the CCs. For those pairs, the
two-pion rejection factor is expected to be at about $10^7$. The
remaining pairs will be discarded from the analysis. This will not have any
significant impact on the statistical uncertainty.

In the photon-energy range of the proposed experiment, the total cross section
for $\pi^+\pi^-$ production is 0.1 mb. With a pion pair rejection factor of
$10^7$, there would be a pion background at the 5\% level if the
total $e^+e^-$ cross section was 0.1 nb. This is comparable to the $J/\psi$
cross section in JLab 12 GeV kinematics. The BH cross section integrated over
$0.5 < Q^{\prime 2} < 7$ GeV$^2$ at $E_\gamma = 11$ GeV is 34 nb. For most
kinematics, and in particular those in the primary range of interest
(\textit{i.e.}, for $4 < Q^{\prime 2} < 9$ GeV$^2$), the contribution to the
total uncertainty from lepton pair misidentification should be small compared
with the statistical uncertainty, and the systematic uncertainty in the
acceptance correction. Performing the measurement with two independent setups,
SoLID and CLAS12, will greatly increase our confidence in being able to
correctly estimate the systematic uncertainties, and in particular those
related to acceptance.
